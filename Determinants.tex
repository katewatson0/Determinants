\documentclass{beamer}

\usepackage[utf8]{inputenc}
\usepackage{tabularx}
\usepackage{amsmath}
\usepackage{amssymb}
\usepackage{booktabs}
\usepackage{centernot}
\usepackage{hyperref}
\usepackage{graphicx}
\usepackage{eufrak}
\usepackage{blindtext}
\graphicspath{{pics/}}

\usetheme{Warsaw}

\title{Determinants}
\subtitle{Group 12}

\author{Catterwell, A. \quad Smith, M. \quad Wang, R. \quad Watson, K.}

\institute{University of Edinburgh}

\begin{document}

\begin{frame}
    \maketitle
\end{frame}

\begin{frame}{Overview}
    \tableofcontents
\end{frame}

\section{Using LU decomposition to compute determinants}

\subsection{What is LU decomposition?}

\begin{frame}{What is LU decomposition?}

    Somebody else do this section

    \begin{itemize}
        \item What LU decomposition is
        \item Its limitations
        \item How PLU decomposition addresses these limitations
    \end{itemize}

\end{frame}

\subsection{How it helps us compute determinants}

\begin{frame}{How it helps us compute determinants}

    some maths.

\end{frame}

\section{Other algorithms for computing determinants}

\subsection{Laplace Expansion}

\begin{frame}{Laplace Expansion}
    The Laplace (1st row) expansion for computing determinants is usually the first method taught for
    computing determinants of $3 \times 3$ matrices and larger.

    \begin{block}{Theorem}
        The formula for the (1st row) Laplace expansion of $A \in \mathrm{Mat}(n, \mathbb{R})$ is given as:
        \[
            |A| = \sum_{j=1}^n a_{1j}\cdot C_{1j}
        \]
        where $C_{ij}$ is the $(i, j)$ cofactor of $A$.
    \end{block}

    Its runtime complexity of $\mathcal{O}(N!)$ is poor.

\end{frame}

\subsection{Leibniz Formula}

\begin{frame}{Leibniz Formula}
    \begin{block}{Definition}
        The Leibniz formula defines the determinant of an $n \times n$ matrix $A$ as follows:
        \[
            |A| = \sum_{\sigma \in \mathfrak{S}_n} \mathrm{sgn}(\sigma) \product_{i=1}^n a_{i \sigma(i)}
        \]
        where $\mathfrak{S}_n$ is the set of permutations length $n$.
    \end{block}

    Computing the determinant using this method is slow with runtime $\mathcal{O}((N+1)!)$
\end{frame}

\subsection{Gaussian Elimination}

\begin{frame}{Gaussian Elimination}

    \begin{itemize}

        \item The determinant of a triangular matrix can be computed by taking the product of its
            diagonal entries (which is a quick $\mathcal{O}(N)$ operation).

        \item Any invertible square matrix can be transformed into echelon form by performing
            Gaussian elimination, which has runtime $\mathcal{O}(N^3)$.

    \end{itemize}
    Conventional Gaussian elimination requires division, meaning that solutions maybe inexact,
    so precision is lost.

    This is addressed by...

\end{frame}

\subsection{Bareiss Algorithm}

\begin{frame}{Bareiss Algorithm}

    \begin{itemize}

        \item Addresses the issue of precision-loss by performing \emph{integer-preserving}
            Gaussian elimination on integer matrices.

        \item The runtime complexity is $\mathcal{O}(N^3)$ which is the same as conventional
            Gaussian Elimination, whilst preserving exactness.

        \item To be continued.

    \end{itemize}

\end{frame}

\subsection{Bird's Algorithm}

\begin{frame}{Bird's Algorithm}
    Define $\mu : \mathrm{Mat}(n, \mathbb{R}) \to \mathrm{Mat}(n, \mathbb{R})$:
    \[
        \mu(X) =
        \begin{bmatrix}{}
            \mu_{2,2} - x_{2,2} & x_{1,2}             & \ldots & x_{1,n-1}           & x_{1,n} \\
            0                   & \mu_{3,3} - x_{3,3} & \ldots & x_{2,n-1}           & x_{2,n} \\
            \vdots              & \vdots              & \ddots & \vdots              & \vdots \\
            0                   & 0                   & \ldots & \mu_{n,n} - x_{n,n} & x_{n-1,n} \\
            0                   & 0                   & \ldots & 0                   & 0
        \end{bmatrix}
    \]
    and $F_A : \mathrm{Mat}(n, \mathbb{R}) \to \mathrm{Mat}(n, \mathbb{R})$, with $A \in \mathrm{Mat}(n, \mathbb{R})$
    \begin{align*}{}
        F_A(X)    & = \mu(X)\cdot A \\
        F_A^2(X)  & = \mu(F_A(X)) \cdot A \\
                  & \vdots \\
        F_A^n(X)  & = \mu(F_A^{n-1}(X)) \cdot A \\
    \end{align*}

\end{frame}

\begin{frame}{Bird's Algorithm (cont.)}

    \begin{block}{Bird's Theorem}
        \[
            F_A^{n-1}(A) = 
            \begin{bmatrix}{}
                d      & 0      & \ldots & 0 \\
                0      & 0      & \ldots & 0 \\
                \vdots & \vdots & \ddots & \vdots \\
                0      & 0      & 0      & 0
            \end{bmatrix}
            \text{with} \ d = 
            \begin{cases}{}
                |A|  & \text{odd} \ n \\
                -|A| & \text{even} \ n \\
            \end{cases}
        \]
    \end{block}
    
    \begin{itemize}

        \item Enables the \emph{division-free} computation of determinants in $\mathcal{O}(n\cdot M(n))$
            where $M(n)$ is the runtime complexity of the matrix multiplication algorithm used.

        \item Given a good matrix multiplication algorithm with runtime $\mathcal{O}(n^{2.376})$ 
            (\emph{Coppersmith-Winograd}),
            this algorithm runs in $\mathcal{O}(n^{3.376})$.

    \end{itemize}

\end{frame}

\section{Summary}

\subsection{A comparison of the determinant algorithms}

\begin{frame}{A comparison of the determinant algorithms}
    \begin{tabular}{l l l}
        \toprule
        \emph{Algorithm}     & \emph{Runtime}           & \emph{Exact} \\
        \midrule
        Laplace Expansion    & $\mathcal{O}(N!)$        & Yes \\
        Leibniz Formula      & $\mathcal{O}((N+1)!)$    & Yes \\
        LU Decomposition     & $\mathcal{O}(N^3)$       & No \\
        Gaussian Elimination & $\mathcal{O}(N^3)$       & No \\
        Bareiss' Algorithm   & $\mathcal{O}(N^3)$       & Yes \\
        Bird's Algorithm     & $\mathcal{O}(N^{3.376})$ & Yes \\
        \bottomrule
    \end{tabular}

\end{frame}

\end{document}
